\documentclass[12pt,a4paper]{article}
\usepackage[utf8]{inputenc}
\usepackage[T2A]{fontenc}
\usepackage[russian]{babel}
\usepackage{amsmath,amsfonts,amssymb}
\usepackage{graphicx}
\usepackage{hyperref}
\usepackage{listings}
\usepackage{xcolor}
\usepackage{geometry}

\geometry{margin=2.5cm}

\lstset{
    basicstyle=\ttfamily\small,
    breaklines=true,
    frame=single,
    backgroundcolor=\color{gray!10}
}

\title{Отчёт по практической работе №3\\
\large Ансамбли алгоритмов для решения задачи регрессии. Веб-сервер}
\author{Практикум 317 группы, ММП ВМК МГУ}
\date{2025}

\begin{document}

\maketitle

\section{Введение}

В данной практической работе были реализованы два классических алгоритма машинного обучения для задачи регрессии: \textbf{Random Forest} (случайный лес) и \textbf{Gradient Boosting} (градиентный бустинг). Также был разработан веб-сервер с REST API и графическим интерфейсом для обучения и инференса моделей.

\section{Реализация алгоритмов}

\subsection{Random Forest (Случайный лес)}

Случайный лес --- это ансамблевый алгоритм, который строит множество независимых деревьев решений и усредняет их предсказания.

\textbf{Основные особенности реализации:}
\begin{itemize}
    \item \textbf{Bootstrap-сэмплирование}: каждое дерево обучается на случайной выборке с возвращением
    \item \textbf{Параметры}: \texttt{n\_estimators}, \texttt{max\_depth}, \texttt{max\_features}
    \item \textbf{Early stopping}: остановка обучения при отсутствии улучшения метрики
    \item \textbf{Сериализация}: сохранение и загрузка обученных моделей через \texttt{joblib}
\end{itemize}

Формула предсказания:
\[
    \hat{y}(x) = \frac{1}{T} \sum_{t=1}^{T} b_t(x)
\]
где $T$ --- количество деревьев, $b_t(x)$ --- предсказание $t$-го дерева.

\subsection{Gradient Boosting (Градиентный бустинг)}

Градиентный бустинг строит ансамбль последовательно, где каждое следующее дерево обучается на остатках (антиградиенте) предыдущих.

\textbf{Основные особенности реализации:}
\begin{itemize}
    \item \textbf{Последовательное обучение}: каждое дерево корректирует ошибки предыдущих
    \item \textbf{Learning rate}: коэффициент сжатия для регуляризации
    \item \textbf{Начальное приближение}: среднее значение целевой переменной
    \item \textbf{Early stopping}: аналогично Random Forest
\end{itemize}

Формула обновления:
\[
    F_m(x) = F_{m-1}(x) + \eta \cdot b_m(x)
\]
где $\eta$ --- learning rate, $b_m(x)$ --- дерево, обученное на остатках $y - F_{m-1}(x)$.

\section{Веб-сервер}

\subsection{Архитектура}

Система построена на микросервисной архитектуре с двумя компонентами:

\begin{enumerate}
    \item \textbf{Backend} (FastAPI) --- REST API на порту 8000
    \item \textbf{Frontend} (Streamlit) --- веб-интерфейс на порту 8501
\end{enumerate}

\subsection{Backend API}

Реализованы следующие эндпоинты:

\begin{table}[h]
\centering
\begin{tabular}{|l|l|l|}
\hline
\textbf{Метод} & \textbf{Endpoint} & \textbf{Описание} \\
\hline
GET & /existing\_experiments/ & Список экспериментов \\
POST & /register\_experiment/ & Создание эксперимента \\
GET & /experiment\_config/ & Получение конфигурации \\
GET & /needs\_training & Проверка статуса обучения \\
POST & /train/ & Запуск обучения \\
GET & /convergence\_history/ & История сходимости \\
POST & /predict/ & Предсказание \\
GET & /health & Проверка здоровья \\
\hline
\end{tabular}
\end{table}

\subsection{Структура проекта}

\begin{lstlisting}
ensembles/              # ML алгоритмы
  random_forest.py      # RandomForestMSE
  boosting.py           # GradientBoostingMSE
  utils.py              # RMSLE, early stopping
  backend.py            # ExperimentConfig schema
  frontend.py           # HTTP Client

backend/                # FastAPI сервер
  ml_app.py             # Точка входа
  src/experiments/      # API модуль
    schemas.py          # Pydantic схемы
    service.py          # Бизнес-логика
    router.py           # Эндпоинты

ui.py                   # Streamlit интерфейс
runs/                   # Сохранённые модели
\end{lstlisting}

\section{Docker}

Приложение упаковано в Docker с использованием \texttt{docker-compose}:

\begin{lstlisting}
services:
  backend:
    build: backend/Dockerfile
    ports: 8000:8000
    healthcheck: /health

  frontend:
    build: Dockerfile.streamlit
    ports: 8501:8501
    depends_on: backend (healthy)
\end{lstlisting}

\textbf{Запуск:}
\begin{lstlisting}[language=bash]
docker-compose up -d
# или
make up
\end{lstlisting}

\section{Функциональность интерфейса}

\begin{enumerate}
    \item \textbf{Создание эксперимента}: выбор модели, гиперпараметров, загрузка CSV
    \item \textbf{Обучение модели}: визуализация кривых обучения (train/val loss)
    \item \textbf{Инференс}: загрузка тестовых данных и получение предсказаний
\end{enumerate}

\section{Заключение}

В ходе работы были успешно реализованы:
\begin{itemize}
    \item Алгоритмы Random Forest и Gradient Boosting с поддержкой early stopping
    \item REST API на FastAPI с полной документацией (Swagger UI)
    \item Веб-интерфейс на Streamlit для работы без знания Python
    \item Docker-контейнеризация для простого развёртывания
\end{itemize}

Система позволяет пользователям без навыков программирования обучать модели машинного обучения и делать предсказания через удобный веб-интерфейс.

\end{document}
